\documentclass[12pt]{article}
\usepackage[margin=0.5in]{geometry}
\begin{document}
    \title{To Do List and Open Issues}
    \author{Javier Villarreal}
    \date{}
    \maketitle

    \section*{To Do}

    \subsection*{Build Options}
    Before development can go fully underway, build and compilation options need to be defines in the makefile. Some options to bear in mind:
    \begin{itemize}
        \item Debug vs Release builds. (Note: If using VSCode, the different builds should be reflected in the JSON option files)
        \item Warning and error flags (\texttt{-Wall}, \texttt{-pedantic}, \texttt{-std=...})
        \item Optimizations, \texttt{-Og} for debugging and \texttt{-O2,-O3} for release.
    \end{itemize}

    \subsection*{Read in simulation metadata}
        The data files necessary to make the code run are different depending on whether the code is running a 2- or 3-dimensional problem. The common files are:
        \begin{itemize}
            \item SimulationValues.txt
            \begin{itemize}
                \item Mach number
                \item Angle of Attack
                \item Reynolds Number
            \end{itemize}
            \item Sizes.txt
            \begin{itemize}
                \item \# of domain nodes (including body boundary nodes)
                \item \# of body boundary nodes
                \item \# of farfield boundary nodes
                \item (Only in 3D code) \# of symmetry nodes
                \item \# of cloud nodes per domain/body boundary node (for differential quadrature)
                \item \# of ghost nodes per body boundary node (for differential quadrature, should equal (cloud-1)/2)
                \item \# of extrapolation nodes per farfield boundary node
                \item \# of total nodes (= domain + farfield + ghost*body)
            \end{itemize}
        \end{itemize}
    
    The current task is to write a subroutine to read those files. The subroutine should reside in a separate module, since file I/O is not inherently a SOMA-specific task.
\end{document}