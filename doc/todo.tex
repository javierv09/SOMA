\documentclass[12pt]{article}
\usepackage[margin=0.5in]{geometry}
\usepackage{multicol}
\begin{document}
    \title{To Do List and Open Issues}
    \author{Javier Villarreal}
    \date{}
    \maketitle

    \section{Active Tasks}

    \subsection{Data architecture and variables}
        The current task is to create all the variables necessary to hold the simulation data. So far, the only existing variables are those used to read in text file data.
        
        Most of the C++ variables were defined as members in classes. The classes are
        \begin{multicols}{3}
            \begin{itemize}
                \item Data
                \item Approximator
                \item SimFluid
                \item FlowVars
                \item Time
                \item Domain
                \item InternalBoundary
                \item ExternalBoundary
                \item GeneticAlgorithm
            \end{itemize}
        \end{multicols}
        Only some classes are instantiated in the main code, though, with some only existing within other classes as members. The data structure is roughly
        \begin{verbatim}
Data is never instantiated, serves as an "abstract" class
    Q[4] => air.[r,u,v,e]
Domain simRegion
    InternalBoundary body
    ExternalBoundary edge
SimFluid air
    FlowVars r,u,v,e
    Time t
Approximator solver
    GeneticAlgorithm ga
    dom => simregion
    air => air
        \end{verbatim}
        where \texttt{=>} denotes pointer variables

    A full listing of the classes and their respective class member variables (not member functions) can be found in \texttt{C++\_classes.txt}. Of course, there are also variables defined within the scope of certain functions. Those are listed in \texttt{C++\_functions.txt}.
    
    \newpage
    \section{Future Improvements}
    \subsection{Build Options}
    Improvement could be made to the makefile (or a different build software) for having different builds in separate directories for debugg and release versions. There are also some softwares that automatically figure out dependencies, so they don't have to be explicitly stated in the makefile.

    \subsection{Pre-processing/Data structure}
    The way the data is stored in the input text files and in variables within the code is suboptimal. Changes could be made both within pre-processing (i.e. the Matlab codes that generate those files) and how the variables are declared in Fortran.

    \newpage
    \section{Completed Tasks}

    \subsection{Build Options}
    The code is built with a basic Makefile. Compilation flags are hardcoded, so a \texttt{make clean} is needed anytime flags are changed.

    \subsection{Read in simulation data}
    The data files necessary to make the code run are different depending on whether the code is running a 2- or 3-dimensional problem. For 2D, geometries, the metadata files are:
    \begin{itemize}
        \item SimulationValues.txt (Mach, AOA, Re)
        \item Sizes.txt (\# of domain, body, farfield, cloud, ghost, extrapolation, total nodes)
    \end{itemize}
    and the geometry data files are:
    \begin{itemize}
        \item x,y.txt (node coordinates)
        \item DX,DY.txt (DQ coefficients)
        \item EC.txt (extrapolation coefficients)
        \item Jd,Jb,Jf.txt (domain, body, farfield node indices)
        \item nxb,nyb.txt (body node unit normal vectors)
        \item nxf,nyf.txt (farfield node unit normal vectors)
        \item s11,s12,s21,s22.txt (Flow tangency matrices)
    \end{itemize}
\end{document}