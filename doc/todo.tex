\documentclass[12pt]{article}
\usepackage[margin=0.5in]{geometry}
\begin{document}
    \title{To Do List and Open Issues}
    \author{Javier Villarreal}
    \date{}
    \maketitle

    \section{Active Tasks}

    \subsection{Data architecture and variables}
        The current task is to create all the variables necessary to hold the simulation data. So far, the only existing variables are those used to read in text file data. The rest of the data is organized like so in the C++ implementation:

    
        There could be room for improvement in the structure -- not necessarily for performance, but for readability of the code.
    
    \newpage
    \section{Future Improvements}
    \subsection{Build Options}
    Improvement could be made to the makefile (or a different build software) for having different builds in separate directories for debugg and release versions. There are also some softwares that automatically figure out dependencies, so they don't have to be explicitly stated in the makefile.

    \subsection{Pre-processing/Data structure}
    The way the data is stored in the input text files and in variables within the code is suboptimal. Changes could be made both within pre-processing (i.e. the Matlab codes that generate those files) and how the variables are declared in Fortran.

    \newpage
    \section{Completed Tasks}

    \subsection{Build Options}
    The code is built with a basic Makefile. Compilation flags are hardcoded, so a \texttt{make clean} is needed anytime flags are changed.

    \subsection{Read in simulation data}
    The data files necessary to make the code run are different depending on whether the code is running a 2- or 3-dimensional problem. For 2D, geometries, the metadata files are:
    \begin{itemize}
        \item SimulationValues.txt (Mach, AOA, Re)
        \item Sizes.txt (\# of domain, body, farfield, cloud, ghost, extrapolation, total nodes)
    \end{itemize}
    and the geometry data files are:
    \begin{itemize}
        \item x,y.txt (node coordinates)
        \item DX,DY.txt (DQ coefficients)
        \item EC.txt (extrapolation coefficients)
        \item Jd,Jb,Jf.txt (domain, body, farfield node indices)
        \item nxb,nyb.txt (body node unit normal vectors)
        \item nxf,nyf.txt (farfield node unit normal vectors)
        \item s11,s12,s21,s22.txt (Flow tangency matrices)
    \end{itemize}
\end{document}