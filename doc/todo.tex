\documentclass[12pt]{article}
\usepackage[margin=0.5in]{geometry}
\begin{document}
    \title{To Do List and Open Issues}
    \author{Javier Villarreal}
    \date{}
    \maketitle

    \section*{To Do}
    \subsection*{Read in simulation metadata}
        The data files necessary to make the code run are different depending on whether the code is running a 2- or 3-dimensional problem. The common files are:
        \begin{itemize}
            \item SimulationValues.txt
            \begin{itemize}
                \item Mach number
                \item Angle of Attack
                \item Reynolds Number
            \end{itemize}
            \item Sizes.txt
            \begin{itemize}
                \item \# of domain nodes (including body boundary nodes)
                \item \# of body boundary nodes
                \item \# of farfield boundary nodes
                \item (Only in 3D code) \# of symmetry nodes
                \item \# of cloud nodes per domain/body boundary node (for differential quadrature)
                \item \# of ghost nodes per body boundary node (for differential quadrature, should equal (cloud-1)/2)
                \item \# of extrapolation nodes per farfield boundary node
                \item \# of total nodes (= domain + farfield + ghost*body)
            \end{itemize}
        \end{itemize}
    
    The current task is to write a subroutine to read those files. The subroutine should reside in a separate module, since file I/O is not inherently a SOMA-specific task.
\end{document}